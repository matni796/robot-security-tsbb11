\section{Results and Evaluation}
In this type of project there is no typical solution which final results can be evaluated by comparing the results with ground truth values. This is simply because there is no ground truth available for this type of problem. Evaluation of the system must then be done by checking if the system works for all possible scenarios. This type of evaluation is not quite desirable since it will not give a result which can be compared to other implementations of the same or similar problem. However, it will give a good understanding if the system works or not and motivates if the system could be used in real life collaboration areas between robots and humans. The following scenarios have been evaluated:

1. A person walks in into the scene, enters the outer security zone, stops and then leaves. This is the most trivial test and will show if the system produces any valuable results. This situation should be tested from all directions and the test person should enter all zones. 

2.  A person walks in into the scene, enters the outer security zone, stops for a long period of time and then leaves. This situation should be tested from all directions and the test persons should enter all security zones.

3. Two persons enter the scenes, enters the outer security zone, stops for a short period of time and then leaves. This situation should be tested from all directions and the test persons should enter all security zones. The path of the two persons should not interfere with each other, that is keeping a clear distance from each other. 

4. Two persons enter the scenes, enters the outer security zone, stops for a short period of time and then leaves. This situation should be tested from all directions and the test persons should enter all security zones. The path of the two persons should interfere with each other, that is at one point keep a very small distance between each other. 

5. A person enters the scene, enters the outer security zone, stops for a short period of time and then leaves. Before the person leaves it leaves an object behind within the outer security zone. 

6. Two persons enter the scene maintaining physical contact (enough for them to appear as one object in the cluster extraction), enters the outer security zone, terminates the physical contact (the cluster extraction should now extract two objects) and then both leave the scene.

7. Two persons enter the scene maintaining physical contact (enough for them to appear as one object in the cluster extraction), enters the outer security zone, terminates the physical contact (the cluster extraction should now extract two objects) and then one person leaves the scene while the other stays in the outer security zone.


8. Multiple persons enters the scene acting like a herd of sheeps (moving randomly without knowledge about the robot). The persons should once in a while enter security zones. This test makes it possible to count the number of correct classifications and this test more or less correspond to a real life situation. 
