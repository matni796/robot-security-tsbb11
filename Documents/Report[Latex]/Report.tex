%----------------------------------------------------------------------------------------------------------
% Robotics safety report
%----------------------------------------------------------------------------------------------------------

\documentclass[10pt,a4paper]{article}
%----------------------------------------------------------------------------------------------------------
% Nödvändiga bibliotek
%----------------------------------------------------------------------------------------------------------

% Sånt man behöver för vanliga svenska dokument (byt tredje radens argument mot swedish istället för english för svenska dokument)
\usepackage[T1]{fontenc} %svenska tecken, åäö egna bokstäver
\usepackage[utf8]{inputenc} %vald teckenkodning
\usepackage[english]{babel} %engelsk dokumentstandard engelska rubriker

\usepackage[none]{hyphenat} %ta bort ordbrytning

% För att bilder och tabeller ska hamna rätt
\usepackage{float}

% Mer för bilderna
\usepackage{graphicx}
%\usepackage{subcaption}
\usepackage{rotating}
\graphicspath{{images/}} %väg till bilderna

% Om man kanske vill räkna lite matte
\usepackage{amsmath}

% Figurpaket
\usepackage{tikz, tikz-3dplot}

\usetikzlibrary{arrows, decorations.markings}
\usetikzlibrary{shapes.geometric}

% För att texten inte ska hoppa in en bit i nya stycken
\setlength{\parindent}{0cm}
\setlength{\parskip}{0.25cm} %avtånd mellan stycken

%Skapa kommandot HRule som gör ett horisontellt streck 0.5 mm tjockt (används i titlepage)
\newcommand{\HRule}{\rule{\linewidth}{0.5mm}}

%\usepackage{subfiles}

%----------------------------------------------------------------------------------------------------------
% Här börjar dokumentet
%----------------------------------------------------------------------------------------------------------

\begin{document}

\begin{titlepage}

\begin{center}   

% Title
{\Large TSBB11 CDIO Project - Images and Graphics }\\[1cm]

\HRule \\[0.4cm]% horisontellt streck
{\huge \bfseries Robotic Safety}\\[0.4cm]
{\Large Technical documentation}
\HRule \\[1.3cm]% horisontellt streck
Version 1.1
\vspace{2 cm}

\includegraphics[width=0.8\linewidth]{liusig}



\vfill

% Bottom of the page

\begin{center}
  {\large Status}\\[1.5ex]
  \begin{tabular}{|*{3}{p{40mm}|}}
    \hline
    Reviewed &  & \\
    \hline
    Approved & &  \\
    \hline
  \end{tabular}
\end{center}
{\large \today}

\pagebreak


        \textbf{\LARGE Participants}

\begin{center}

  \begin{tabular}{|*{2}{p{40mm}|}}
  \hline \bf Name &  \bf e-mail\\
    \hline
 Olle Fridolfsson&ollfr940@student.liu.se\\
 		\hline
		Niklas Hansson&nikha310@student.liu.se\\
		\hline
		Patrik Hillgren&pathi747@student.liu.se\\
		\hline		
		Benjamin Ingberg &benin542@student.liu.se\\
		\hline
		Pär Lundgren&parlu048@student.liu.se\\
		\hline
		Mattias Nilsson&matni796@student.liu.se\\
		\hline
  \end{tabular}
\end{center}

{\footnotesize
\vspace{0.5\baselineskip}
\textbf{Homepage}: \href{http://www.isy.liu.se/edu/projekt/bildbehandling/2013/robot\_safety/}{http://www.isy.liu.se/edu/projekt/bildbehandling/2013/robot\_safety/} \\
\vspace{1\baselineskip}

\textbf{Customer}: Rickard Olsen, Link\"{o}ping University, Industriell produktion, IEI \\
\textbf{Customer contact}: richard.olsen@liu.se \\
\textbf{Supervisor}: Johan Hedborg, Link\"{o}ping University, johan.hedborg@liu.se \\
\textbf{Examiner}: Michael Felsberg, michael.felsberg@liu.se \\
}

\end{center}
\end{titlepage}

% Preamble

\section*{Preamble}

We would like to thank Johan Hedborg and Rickard Olsen for guidance and support throughout the project.  Special thanks go to Eric Marcil at Yaskawa America Inc. for the help with configuring the controller, Anders Leopold at Yaskawa Nordic AB for hints and directions and to all the people providing answers to us at ROSAnswers\cite{RosAnswers}.
	% Förord, tack osv.
\pagebreak
\tableofcontents
\newpage

% Background - Despription of why this project is performed.
\section{Background}

% Closer description of poject

\section{Project Description}
Robots like SiA20 Motoman are powerful and can in some situations cause harm to humans working in the same area. The goal of this project is to create a system which ensures a safe environment in a human robot collaboration area. This will be done by determining the distance between the closest moving object and the robot and retrieve a control signal to send to the robot depending on that information. The control signal should determine four states:
\begin{enumerate}
\item The closest moving object is outside safety zone 2 (see figure below). Objects are not in the collaborative area, the robot can work at standard motion. 

\item The closest moving object is within safety zone 2. This means that the robot motion shall be reduced since the moving object is within the collaborative area.

\item The closest moving object is within safety zone 1. This means that the robot must stop.

\item The closest moving object is within the emergency zone. This means that the robot must perform an emergency stop, which differs from the usual stop. 
\end{enumerate}

\begin{figure}[H]
\begin{center}
\includegraphics[width=4 cm]{robotsafetyzone}
\caption{Flowchart of tracking algorithm}
\label{robotsafetyzone}
\end{center}
\end{figure}


% Description on the libraries on environment used.

\section{Development Environment}

\subsection{Robotic Operating System (ROS)}
The implementation is done in ROS-packages according to ROS-standards. ROS uses sockets (topics) to send data between different programs (nodes). The nodes and topics are very useful since it gives the opportunity to run the system partitioned in different programs. The main advantages of using ROS is the possibility of simple communication with the robot. ROS also provides information about the whereabouts of the robot which in this system is of high interest. Another advantage of ROS is that it contains many third party libraries, common for projects including robots and computer vision, such as OpenCV and Point Cloud Library.  

\subsection{External libraties}
OpenCV, Point Cloud Library(PCL).

% Overview of how the programm is structured

\section{System Overwiev}



\section{Subsystems}

% Input data 
\subsection{Input Data}
\subsubsection{Joint states of robot}
When connection with the controller is established the joint states will be available. Each joint has a value in radians describing its state, i.e. its revolution. The objective is to build a model of the robot in camera centered coordinates where the distance to a moving object, represented as a point cloud in 3D space can be computed.
The data regarding joint states is used for both visualization, in the shape of a mesh figure, and for calculation of the robot's position in relation to other objects. These two are independent with respect to each other.

\subsubsection{Depth-image from Kinect}
The IR projector and IR camera provides a depth image of the collaboration area. The data is used for background segmentation and calibration. 

\begin{figure}[H]
\begin{center}
\includegraphics[width=12 cm]{screenshot_depth_image}
\caption{Depth image}

\end{center}
\end{figure}

\subsubsection{RGB-image from Kinect}
The RGB camera is used for calibration of both intrinsic and extrinsic camera parameters. From those it is possible to determine a relationship between the camera and a well known positioned calibration pattern. 

\begin{figure}[H]
\begin{center}
\includegraphics[width=12 cm]{image_color}
\caption{RGB-image}

\end{center}
\end{figure}

\subsubsection{tf}
The tf subsystem is a transformation management system available in ROS which is written and maintained by Tully Foote. Since the system will have dozens of frames the tf subsystem traverser the set of known frame relations to give the transformation parameters between two arbitrary reference frames at any point in time. 

% Calibration between coordinate systems.

\subsection{Calibration}
To relate the camera system to the world system the system has to be calibrated. For calibration of the camera intrinsic parameters and distortions from the pinhole camera model the system uses the models and methodology described by Zhang, 2000 \cite{zhang}. This is performed offline.

For online calibration of the camera extrinsics the same distortion parameters are used with a world fixed calibration pattern.

The online calibration is very computationally expensive however since the system is supposed to be stationary recalibration is only done once a second to compensate for small adjustments on the system.

Most of the building blocks of the calibration is implemented by the OpenCV library \cite{camcal}. 

\subsubsection{Intrinsic calibration}
Before image distortions the projection of the 3D-world to the image plane is described by this matrix:

\[ \left( \begin{array}{ccc}
a & b & c \\
d & e & f \\
g & h & i \end{array} \right)\] 

Where $f_x$ and $f_y$ are the number of pixels per unit of length, cx and cy are the center pixel coordinates of the image. X, Y and Z are the camera relative positions of a visible point with x, y and w as the homogeneous representation of the pixel coordinate where the point is projected.

Distortions from this model is modelled as radial distortion as well as tangential distortions. Radial distortions produce a fish-eye effect on the image and is compensated for with this model:

$radial_corrected_x = f(x)$
$radial_correcced_y = f(y)$

Tangential distortions due to the lense not being perfectly aligned is compensated with this model:

$tangential_corrected_x = g(x)$
$tangential_corrected_y = g(y)$

Since this is a standard model external libraries have support for these parameters and the error correction from calculating the distortion parameters will cascade into other functions. Intrinsic and distortion calibration is done offline and the parameters are saved into a YAML file.

\subsubsection{Extrinsic calibration}
To relate the position and state of the robot to the camera there needs to be a reference between a fixed robot frame and the camera frame. To solve this a calibration pattern is placed at a known position relative to the robot.

This calibration pattern is then detected in the image and the solution to the PnP (Perspective-n-Point) problem with the camera parameters is used as extrinsic parameters.

To minimize noise and increase robustness a large chessboard pattern with 6x8 known points is used for calibration. However there is no known analytical solutions to the PnP problem for n > 3 so the problem is solved using optimization (minimization of the sum of squared reprojection errors) with the Levenberg-Marquardt iterative optimization algorithm.

To prevent the optimization getting stuck in local optima the optimization is done on an analytical solution to the P3P problem. The points used for the analytical solution are the four corners of the chessboard, three to solve the P3P problem and one to validate which of the four possible solutions is consistent with the rest of the data.

The solution is then further median filtered on one of the rotation parameters to prevent outliers. Though running the system has so far yet to produce an outlier after the previous errors these median filtered points could in the future be averaged to further increase precision if the system demands it.

\subsubsection{Transformation of robot joints}
While running warehouse\_viewer the the transformations between the joints can be obtain given the relation of the joints specified in the robot’s URDF file, see appendix. The tf package included in ROS makes it possible to transform the robot and its joints into a given coordinate system, provided by the RGB camera.

\subsubsection{Transformation into camera coordinate system}
A static transformation between calibration pattern and the base of the robot in world coordinates is set. Given the transformation from camera to world coordinates it is possible to transform each joint to the camera coordinate system. This results in a set of joints on which the distance calculation to moving objects in the scene can be performed. 


\subsection{Background segmentation}

The background segmentation node is based on the backgroundSubtractorMOG2 from OpenCV \cite{BGS} which uses Gaussian mixtures to perform the segmentation. The background segmentation results in a binary image of the foreground. 

\begin{figure}[H]
\begin{center}
\includegraphics[width=12 cm]{screenshot_foreground_image}
\caption{Foreground from background segmentation}

\end{center}
\end{figure}

The foreground and depth image are used to construct a point cloud. For each foreground pixel the corresponding depth value $P_z$, pixel coordinates $P_x$, $P_y$ and intrinsic parameter can be used to obtain the world coordinates as follows.

$P_x = (p_x - c_x) * P_z / f_x$
$P_y = (p_y - c_y) * P_z / f_y$

and the point $P = [P_x, P_y, P_Z]^T$

By applying these formulas to all foreground pixels the point cloud is achieved.




% Clustering 

\begin{frame}{Clustering}
	\begin{itemize}
	\item Euclidian clustering algirithm
	\begin{itemize}
	\item KD-Tree representation
	\item Points which are close are concantenated 
	\end{itemize}
	\end{itemize}
\end{frame}


\subsection{Removal of Robot}

In the background segmentation the system finds everything that is classified as foreground, including the robot. This results in a point cloud that might contain values corresponding to the robot. The purpose of this project is to check if detected objects are within a certain distance from the robot. If the robot itself is one of the objects the distance will always be minimal and the system will not work. This means that the system needs to remove clusters corresponding to the robot from the other clusters before it can calculate distances and make a decision. 

Removing the clusters corresponding to the robot is done by using the joint coordinates of the robot. Since the system only knows the joints position and not the entire hull of the robot the first thing that needs to be done is to create a convex hull corresponding to an approximation of the real hull of the robot. The approximated hull is achieved by creating lines between connected joints and applying a cylinder around these lines. The radius of the cylinder is given by the greatest radius of the robot between these joints. In other words the cylinders corresponding to the hull of the robot will cover up the whole robot. After the hull is created a simple check is performed which finds out if a certain amount of the points in a cluster are inside any of these cylinders. If this is the case these clusters are excluded for further processing. It is not trivial to know if a point is inside a cylinder, the used method is copied from Greg James. \cite{cylinder} The pseudo code for the algorithm is described below.

point 1= the point that is checked if it is inside a cylinder
point 2= the first point of the cylinder 
point3=the second point of the cylinder
vector= the vector from point 2 to 3.
%1. Translate point1 so that point2 lies in the origin.
%2. Check if the dot product point1*vector is positive and smaller than |vector|^2
%3. If it is not, it is outside the cylinder horizontally.
%3. Calcu


\begin{figure}[H]
\tdplotsetmaincoords{60}{120}
\begin{tikzpicture}
	[scale=5,
		tdplot_main_coords,
		axis/.style={->,black,very thick},
		vector/.style={-stealth,black,very thick},
		vector guide/.style={dashed,black,thick}]

	%standard tikz coordinate definition using x, y, z coords
	\coordinate (O) at (0,0,0);
	%\coordinate (v) at (1,1,1.5);
	%\coordinate (c1) at (0.1,0.1,0);
	%\coordinate (c2) at (1.5,1.5,0.5);
	
	%tikz-3dplot coordinate definition using r, theta, phi coords
	\tdplotsetcoord{v}{1.5}{55}{55}
	\tdplotsetcoord{c2}{1.5}{90}{55}
	%draw axes
	\draw[axis] (0,0,0) -- (1,0,0) node[anchor=north east]{$x$};
	\draw[axis] (0,0,0) -- (0,1,0) node[anchor=north west]{$y$};
	\draw[axis] (0,0,0) -- (0,0,1) node[anchor=south]{$z$};
	
	%draw a vector from O to P
	\draw[vector] (O) -- (v);
	\draw[vector] (O) -- (c2);
	\draw[vector guide] (v) -- (vxy);
    \draw[vector] (O) -- (vxy);
	%\draw[vector guide] (c2) -- (vxy);
	
	%\draw (v) circle[radius=2pt];
	\fill (v) circle[radius=0.15mm];
	\fill (vxy) circle[radius=0.15mm];

	\fill (c2) circle[radius=0.15mm];
	
\node[above right] at (vxy) {\hspace{5 pt}$\mathbf{v_1 = v^Te\cdot e}$};
\node[right] at (v) {\hspace{10 pt}$\mathbf{v=p-c_1}$};
\node[right] at (c2) {\hspace{10 pt}$\mathbf{e=c_2-c_1}$};

\end{tikzpicture}
\caption{Geometry for computing the orthogonal distance to a vecor}
\label{geometry}
\end{figure}
\vspace{20 pt}


The objective is to conclude if {\bf p} is inside a cylinder with squared radius $r^2$ and endpoints ${\bf c_1}$, ${\bf c_2}$. First ${\bf c_1}$ is subtracted from all points to get the vectors shown in figure \ref{geometry}. $\bf v_1$ is obtained by projecting $\bf v$ onto $\bf e$ The condition that ${\bf v}$ is in between the caps (endpoints) of the cylinder is the following:\\
 $ 0 < {\bf v}^T{\bf e} < \| {\bf c_1}-{\bf c_2}\|$ if this is not fulfilled this distance is discarded. Otherwise the orthogonal squared distance using pythagoras formula is\\
$\displaystyle d^2 = \|{\bf v}\|^2 - \|{\bf v_1}\|^2= {\bf v}^T{\bf v} -\frac{({\bf v}^T{\bf e})^2}{{\bf e}^T {\bf e}}$ if $d^2 < r^2$ the point is inside the cylinder



\subsection{Distance Calculation}
For each cluster of points extracted, the remaining problem is to find the smallest distance to the  robot. This is done by brute force, the system iterates through all clusters and for each point it finds the closest euclidean distance to the robot. The data which is saved is the distance itself, the point in the cluster and the joint of the robot which corresponds to the closest distance. The reason for saving the closest point in the cluster and the robot joint is to make it possible to visualize the closest point-joint pair. To gain performance the distance calculation is performed simultaneously as the system is removing parts that correspond to the robot.

% Tracking of Objects

\section{Tracking}


\subsection{Visualization}

The results of the system is visualized in rviz, which is a 3D visualization tool for ROS. 

\subsubsection{Robot}
The data published by the controller in JOINT\_STATES are the radial position of the joints. The warehouse\_viewer will use these states transformed into the coor The URDF file specifies the links between the robot’s joints and the mesh files (.stl) and tf will contain the transformations between the joints which are used in warehouse\_viewer to create 3D model of the robot. 

\begin{figure}[H]
\begin{center}
\includegraphics[width=12 cm]{robot}
\caption{3D-model of robot.}

\end{center}
\end{figure}

\subsubsection{Objects}
The objects in the image that the tracking algorithm considers being real moving objects are visualized. They are published as point clouds so that rviz can subscribe to them and visualize them. If the tracking algorithm believes that an object has disappeared because of non-movement, it publishes the last seen point cloud of the object i.e. static objects in the scene are objects that the system thinks are standing still. 

\begin{figure}[H]
\begin{center}
\includegraphics[width=12 cm]{humanandrobot}
\caption{Human beside the robot.}

\end{center}
\end{figure}

\begin{figure}[H]
\begin{center}
\includegraphics[width=12 cm]{humanandrobotinworld2}
\caption{Humans and robot placed in world coordinate system.}

\end{center}
\end{figure}  

\subsubsection{Distances}
To visualize the resulting state of the system (the current safety zone) a line between the closest object and the closest joint of the robot is drawn. This line switches color depending on which state the system is in. The color of the line is thereby the result of the system. 

\begin{itemize}
  \item Red - Emergency zone
  \item Yellow - Safety zone 1 
  \item Green - Safety zone 2
  \item No line - Outside safety zone 2 
\end{itemize}

//lägg in en bild på hela systemet



\section{Results and Evaluation}
In this type of project there is no typical solution which final results can be evaluated by comparing the results with ground truth values. This is simply because there is no ground truth available for this type of problem. Evaluation of the system must then be done by checking if the system works for all possible scenarios. This type of evaluation is not quite desirable since it will not give a result which can be fully compared to other implementations of the same or similar problem. However, it will give a good understanding if the system works or not and motivates if the system could be used in real life collaboration areas between robots and humans. The following scenarios have been evaluated:

1. A person walks in into the scene, enters the outer safety zone, stops and then leaves. This is the most trivial test and will show if the system produces any valuable results. This situation should be tested from all directions and the test person should enter all zones. 

2.  A person walks in into the scene, enters the outer safety zone, stops for a long period of time and then leaves. This situation should be tested from all directions and the test persons should enter all safety zones.

3. Two persons enter the scenes, enters the outer safety zone, stops for a short period of time and then leaves. This situation should be tested from all directions and the test persons should enter all safety zones. The path of the two persons should not interfere with each other, that is keeping a clear distance from each other. 

4. Two persons enter the scenes, enters the outer safety zone, stops for a short period of time and then leaves. This situation should be tested from all directions and the test persons should enter all safety zones. The path of the two persons should interfere with each other, that is at one point keep a very small distance between each other. 

5. A person enters the scene, enters the safety zones, stops for a short period of time and then leaves. Before the person leaves it leaves an object behind within the safety zones. 

6. Two persons enter the scene maintaining physical contact (enough for them to appear as one object in the cluster extraction), enters the outer safety zone, terminates the physical contact (the cluster extraction should now extract two objects) and then both leave the scene.

7. Two persons enter the scene maintaining physical contact (enough for them to appear as one object in the cluster extraction), enters the outer safety zone, terminates the physical contact (the cluster extraction should now extract two objects) and then one person leaves the scene while the other stays in the outer safety zone.

8. Multiple persons enters the scene acting like a herd of sheep(moving randomly without knowledge about the robot). The persons should once in a while enter security zones. This test makes it possible to count the number of correct classifications and this test more or less correspond to a real life situation. 

\subsection{Performance of system using tracking functionality}
A person can enter the area, have its distance to the robot visualized with either a green, yellow or red marker.
The marker is illustrated in form of a line between the robot and the object.
The marker might switch between two joint that are approximately on the same distance or between the knees of a person. 

The tracking algorithm performs very well when a single object is within the collaborating area.
The tracker stays with the object and holds on to it, even when it is no longer visible in the foreground. 
Consequently the tracking algorithm seems to work very good for the evaluation of scenario 1 and 2.

For scenario 5 the tracking will not register the objects left by the person since it is not possible for objects to appear inside the safety zones. However, if the object is too big there is a chance that the system will start to track the object instead of the person. 

When two or more objects are within the collaborating area the system gets choppy.
The current computer does not manage to process all data in real time and this produces negative artefacts.
When two or more objects are within the collaborating area there is a risk that objects move too far between two concurrent frames.
If a person moves further than what is allowed between two frames the tracker will think that the previous object melted into the background and that a new object has appeared.
The system updates the collaborating area with a very low frequency and the displacement could therefore be more than what the tracker tolerates.
The tracker has a maximum length for allowed displacement of an object between two concurrent frames.
This length could be increased but that would also increase the risk of losing an object between to frames.
E.g., since the tracker looks for a similar object within this displacement distance it might match one person with his friend in the next frame and the person himself could then be in a state where the person is lost. 
This is because it is not possible for an object to appear out of nowhere within a certain distance.\\

Since the tracking starts to fail when there are too many objects in the scene it is difficult to make any conclusions from the evaluation of scenario 3,4,6,7 and 8.

\subsection{Performance of system without tracking}
When tracking is disabled the system makes use of only background subtraction to find objects.
This enables more than one person to be inside the collaborating area since each object then is independent of its location in the previous frame. Since the background modelling works very good with very little noise this method will work rather good in many cases

The problem one might face using this method is that objects are lost when they melt into the background.
If an object is within any of the safety zones the system would simply forget it and the robot would operate in normal speed, higher than what is allowed.
This could be avoided if the system would required that an object would exit each safety mode in correct order!

Using the system without tracking works well in scenario 1,3,4,6,7 and 8. For scenario 2 the system will fail. For scenario 5 the system will detect an object but it will soon starts to disappear.
	
\section{Conclusion and Discussion}
Problems with compatibility with different versions of ubuntu and ROS.  

\subsubsection{Discussions}
This system is based on a background segmentation to find moving objects in the scene which is used to extract points in the depth image. This is only one way of solving this, another more simple way of doing it would be to check at the depth image at once and extract information from it. For instance if the collaboration area would only consist of the floor and a robot objects would be found by thresholding the depth value in the image, that is removing what is floor in the image. This solution one might find more intuitive and simple but it does not take the movement of objects in the scene into account. Another disadvantage when having a background segmentation determining what to use in the image is that there are sometimes parts of objects that are misclassified as background. This problem makes the system more inaccurate and can make a distance between object and robot vary. In an area where there are objects, such as walls, visible for the camera one might prefer using a background segmentation since objects that are not moving will be removed and therefore the calculations needed will reduce significantly.  

One option in the system is to turn off tracking. If it is guaranteed that no one will approach the robot and then stand still for a long time an alternative can be to use the background modelling without the tracking.

In this project it has only been one kinect camera in use, this means that the position of the camera is of high importance since it needs to cover up the entire collaboration area. Unfortunately this was not possible to achieve and the camera setup for this project was therefore not perfect. The resulting camera position was not exactly over the robot, meaning that the system received a tilted view of the scene. A tilted view of the scene can result in objects which in fact are not connected to each other to be interpreted as connected, that is one object instead of two. In a more ideal case, it would be preferred to use more than one camera or have the camera placed exactly above the robot. 

\subsubsection{Uncertainty of calibration}
There is an uncertainty in the accuracy of the calibration. While the computed relation between the known calibration pattern and the camera is based on very robust methods the static calibration between the robot and the pattern is manually measured. Even minor angular errors can give large discrepancy between the robots frame of reference and the cameras frame of reference. To mitigate this the calibration pattern was placed as flat as possible relative to the ground, since this angle is easier to verify than other angles (using e.g. a spirit level). This angle is not optimal from the perspective of detecting the pattern and therefore a rather large calibration pattern is used to ensure detection.

In a more optimal situation the pattern would be painted on something which is mechanically forced into a well known high precision static transform to the robot.

An example of such a system would be to use painted QR codes on different known spots with a large relative distance between them. Since QR-codes are designed to be easily detected with cameras and can convey an orientation as well as an identifier then the system could recalibrate as long as it can see at least one of them (though for the sake of robustness it probably shouldn’t unless it can see four of them). The feasibility of such a system is however out of the scope of this project.



%----------------------------------------------------------------------------------------------------------
% Referenser
%----------------------------------------------------------------------------------------------------------
\begin{thebibliography}{9}

\bibitem{kurssida}
Course homepage for TSBB15, found at URL:
http://www.cvl.isy.liu.se/education/undergraduate/tsbb15/3d-reconstruction-project

\end{thebibliography}

%----------------------------------------------------------------------------------------------------------
% Slut på dokument ! :)
%----------------------------------------------------------------------------------------------------------

\end{document}