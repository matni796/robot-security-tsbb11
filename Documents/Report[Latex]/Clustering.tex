%Clustering of point cloud

\subsection{Clustering}
The clustering node receives a point cloud from the background segmentation node corresponding to the extracted foreground i.e. moving objects. The purpose of the clustering node is to divide the entire point cloud into individual objects. It also removes noise and objects that are not considered to be large enough.  

The reason why individual objects are necessary is because the closest distance to every object compared to the robot is desired. It is not sufficient to know the distance from the closest point in the whole point cloud to the robot. Separating the entire point cloud into individual clouds also provides the possibility of tracking point cloud objects. 


The separation of objects is done by separating the entire point cloud into clusters. This is done using an euclidean cluster extraction algorithm. A simple outline of the algorithm:


\begin{algorithm}[H]
 %\SetLine % For v3.9
 \SetAlgoLined % For previous releases [?]
 \KwData{Point cloud, Cluster parameter}
 \KwResult{Clustering }
 initialization\;
 \For{every point not in a cluster}{
	\For{every other point}{
	check distance to point\; 
  \eIf{distance < Cluster parameter}{
  assign to the same cluster;
 }{
  assign to a new cluster\;
 }
 }
 }
 \caption{How to write algorithms}
\end{algorithm}

\begin{enumerate}
\item for each point
\item check distance to all other points
\item if a point is closer than parameter
\item assign this point to the same cluster
\item if no matching point is found
\item assign to a new cluster
\end{enumerate} 

 The cluster extraction is based on the algorithm given by Point Cloud Library \cite{CE}.
