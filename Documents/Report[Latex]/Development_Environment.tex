% Description on the libraries on environment used.

\section{Development Environment}
The implementation is done in a Linux environment and written in C++.

\subsection{Robot Operating System (ROS)}
The implementation is done in ROS-packages according to ROS-standards. ROS \cite{ROS}
uses sockets (topics) to send data between different programs (nodes). The ROS-version used in this project is ROS Groovy which is suited for a computer running Ubuntu version 12.04. The nodes and topics are very useful since it gives the opportunity to run the system partitioned in different programs. The main advantages of using ROS is the possibility of simple communication with the robot. ROS also provides information about the whereabouts of the robot which in this system is of high interest. Another advantage of ROS is that it contains many third party libraries, common for projects including robots and computer vision, such as: 

\begin{itemize}
\item OpenCV
\item Point Cloud Library (PCL)
\end{itemize}
