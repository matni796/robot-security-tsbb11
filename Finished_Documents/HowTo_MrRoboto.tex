\documentclass[10pt,a4paper]{article}
%You comment with the modulo-sign. Before the \begin keyword you put what packages
%you want to use etc. If you want to include graphics for example you need the graphicX-package
\usepackage{graphicx}
\usepackage{amsmath}
\usepackage[utf8]{inputenc}
\usepackage{float}
\usepackage[none]{hyphenat}
\usepackage{hyperref} % göra länkar

\begin{document}

\section*{MrRoboto}

\section{How to setup system}

\subsection{Initial computer settings}

\begin{enumerate}
\item Install ubuntu 12.04 on your computer.
\item Install ROS GROOVY according to: \url{http://wiki.ros.org/groovy/Installation/Ubuntu}
\item Install git, type:\\
\textit{
\$ sudo apt-get install git} (in a terminal window)\\
Password is \textit{ubuntu} for the computer in the robot lab.
\item Clone the repository of the project.
cd* into a folder where you want the folder containing the code.
Type:\\
\textit{\$ git clone git@github.com:matni796/robot-security-tsbb11}\\
(*cd is a terminal command. If your not familiar with navigation in terminal, see \url{http://linuxcommand.org/lc3\_lts0020.php})

\item Source your own created packages in .bashrc:
	type: gedit ~/.bashrc
add the line:
source \textit{wherever\_path\_you\_put\_repository/robot\-security\-tsbb11/catkin\_ws/src/devel/setup.bash}

\item Install ROS, type: \\
\textit{\$ sudo apt-get install ros-groovy-desktop-full}

\item Install freenect, type: \\
\textit{
\$ sudo apt-get install ros-groovy-freenect-stack\\
\$ sudo apt-get install ros-groovy-freenect-launch
}
\item Disable gspca kernel according to:
	\url{http://openkinect.org/wiki/Getting\_Started}


\item cd into the installed folder and into catkin\_ws and type catkin\_make to build.
\end{enumerate}

\subsection{Setup of DX100 controller}

\begin{enumerate}

\item Make sure the controller’s version supports MotoPlus applications, should be a version ending with -14.
 Current version (20131206) is DS3.53.01A-14. 

\item Load parameters.
The file ALL.PRM can be found in git repo, under the folder robot/. Transfer it to a cf card or usb.
Start the controller regularly.
Go into management mode (see below).
Then go to:\\\\
 EX MEMORY $\rightarrow$ FOLDER, set folder where you put ALL.PRM.\\\\
EX MEMORY $\rightarrow$ LOAD $\rightarrow$ PARAMETERS $\rightarrow$ BATCH PARAMETERS ALL.PRM

You might need to do a safety reset of the flash device. 
This is done by starting up the controller in Maintenance mode.   
In Maintenance Mode, enter Management mode.
INITIALIZE $\rightarrow$ system flash safety reset, might take a while, wait for beep.
Shut off controller and restart it regularly.


NOTE: Might change the setup of the controller. Should be done with caution. Contact 	
Yaskawa if uncertain.
 

\item To install MotoPlus application, follow the tutorial on \url{http://wiki.ros.org/motoman\_driver/Tutorials/InstallServer.}
\begin{itemize}

\item For step 3 in tutorial, see PDF file \textit{MotoPlus Application Installation} in folder \textit{robot/}. 
Follow the instructions on step 2.1. 
The application file: \textit{MpRosSia20.out} to be loaded is also in the in the \textit{robot/} folder.\\

NOTE: We have used an .out file hardcoded for a SIA20D robot. This version does not come with the motoman files from ROS.\\

\item For step 4 in tutorial, transfer \textit{INIT\_ROS.JBI} to CF card or USB device. 
File can be found under \textit{../catkin\_ws/src/motoman/motoman\_driver/Inform/DX100/}. 
Start the controller regularly. In menu, go to: \\\\
Ex MEMORY $\rightarrow$ FOLDER move to the folder where you put \textit{INIT\_ROS.JBI} file. Then:\\\\
Ex Memory $\rightarrow$ LOAD $\rightarrow$ job.
\end{itemize}


\end{enumerate}

\subsection{Setup of computer network settings}

\begin{enumerate}

\item Connect the computer to the controller via ethernet cable. Use the output CN104 on the YCP01 board.
\item Edit settings for wired/local network on computer. Set computer wired network address to 192.168.255.9 and netmask 255.255.255.0.\\


NOTE: On the computer, running Ubuntu 12.04, in the robot lab, there is a profile for the connection named “DX100”. When connecting the controller to the computer via ethernet, choose this.
\end{enumerate}

\section{Run program}
When the setup is done follow these instructions to run the system.
\subsection{Establish connection between controller and computer}
Follow the instructions on \url{http://wiki.ros.org/motoman\_driver/Tutorials/Usage}.
\begin{itemize}


\item On step 3, enter the controller IP \textbf{192.168.255.1}.

\item  On step 3.1.2.1, use the path \textit{..catkin\_ws/src/motoman/motoman\_config/cfg/sia20D\_mesh.xml}\\
\end{itemize}
NOTE: Currently it is not possible to send motion commands to the robot.
This is due to developers lack of implementation skills.


\subsection{Launch system}
While still running \textit{roscore} and the robot node. Run the following comands, each in a seperate terminal.
Use \textit{Ctrl+Shift+T} to add a tab to existing terminal window.
\begin{itemize}
\item \textit{\$ roslaunch freenect\_launch freenect.launch}

\item \textit{\$ rosrun background\_modelling background\_modelling}

\item \textit{\$ rosrun clustering clustering}

\item \textit{\$ rosrun calibration calibration}

\item \textit{\$ rosrun distance\_calc distance\_calc}

\item In order to set the static transform between the calibration pattern and the robot base. In terminal, type:\\
 \textit{\$ rosrun tf static\_transform\_publisher x y z yaw pitch roll base\_link pattern 100}
 
where x, y, z and yaw, pitch, roll has to be specified. With the setup in our project they were: $\left[0, 0, 0, \pi, \pi, \pi \right].$ This might need to be tweaked to get a good calibration.

\item To obtain visualization, type:\\
\textit{\$ roslaunch sia20d\_mesh\_arm\_navigation planning\_scene\_warehouse\_viewer\_sia20d\_mesh.launch}
This should launch the visualisation environment RViz.

\end{itemize}
NOTE: In each terminal there will be data printed about that function current state.

\subsection{Visualization}
RViz provides a wide range of possibilities for visualization.
To begin with there are some fundamental settings that always needs to be performed.
\begin{itemize}

\item Under Global Options, choose /base\_link as Fixed Frame.

\item Add a PointCloud2 and choose /objects as Topic.

\item For one of the Markers choose the Topic /closest\_line
\end{itemize}


\end{document}
